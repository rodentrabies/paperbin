\documentclass{beamer}

\usepackage{amsmath}
\usepackage[style=alphabetic,url=true]{biblatex}
\usepackage{environ}
\usepackage{geometry}
\usepackage{graphicx}
\usepackage{amsmath}
\usepackage{amsfonts}
\usepackage{amssymb}
\usepackage{calrsfs}
\usepackage{listings}
\usepackage{tikz}
\usepackage[T2A]{fontenc}
\usepackage[utf8]{inputenc}
\usepackage[cache=false]{minted}

\graphicspath{ {./graphics/} }
\usetikzlibrary{shapes.arrows,chains}
\usecolortheme{beaver}
\setbeamertemplate{itemize item}[circle]
\setbeamertemplate{itemize subitem}{--}
\addtobeamertemplate{navigation symbols}{}{
  \usebeamerfont{footline}
  \usebeamercolor[fg]{footline}
  \hspace{1em}
  \insertframenumber/\inserttotalframenumber
}
\setminted[Lisp]{
  fontsize=\scriptsize
}
\BeforeBeginEnvironment{minted}{\medskip}
\AfterEndEnvironment{minted}{\medskip}



\title{
  Common Lisp and Introduction to Functional Programming \\
  Lecture 3: Common Lisp Functions and Variables
}
\author{Yuri Zhykin}
\date{Feb 17, 2021}

\begin{document}

\frame{\titlepage}

\begin{frame}[fragile]
  \frametitle{Function Definitions}
  \begin{itemize}
  \item Functions in Common Lisp are defined with \mintinline{Lisp}{defun} form:
\begin{minted}{Lisp}
  (defun <function-name> <parameter-list> <docstring>? <body>)
\end{minted}
  \item Example:
\begin{minted}{Lisp}
  (defun greet (name formally?)
    "Print a greeting for the given `name`, either formal
or informal one, depending on the value of `formally?`."
    (if formally?
        (format t "Good morning, ~a.~%" name) 
        (format t "What's up, ~a!~%" name)))

  (greet "Mr. Holmes" t)
  ;; Good morning, Mr. Holmes.
  (greet "Mike" nil)
  ;; What's up, Mike!
\end{minted}
  \end{itemize}
\end{frame}

\begin{frame}[fragile]
  \frametitle{$\lambda$-lists 1/4}
  \begin{itemize}
  \item Parameter lists in Common Lisp function definitions are called
    $\lambda$-lists (or lambda-lists).
  \item Regular symbols in $\lambda$-lists represent required parameters -
    missing required arguments cause an error to be signaled:
\begin{minted}{Lisp}
  (defun f (x y z) ; x, y, z are parameters
    (* x (+ y z)))

  (f 1 2) ; 1, 2 are arguments for x, y; argument for z is missing
  ;; Error: f got 2 args, wanted 3 args.
\end{minted}
  \end{itemize}
\end{frame}

\begin{frame}[fragile]
  \frametitle{$\lambda$-lists 2/4}
  \begin{itemize}
  \item \textbf{Optional parameters} are parameters that have default value and
    do not have to be provided.
  \item Optional parameters are positional: in order to provide an optional
    argument, all previous ones have to be included.
  \item Optional parameters are separated from the required ones with the special
    symbol \mintinline{Lisp}{&optional}:
\begin{minted}{Lisp}
  (defun greet (name &optional (formally? t) (time-of-day "morning"))
    (if formally?
        (format t "Good ~a, ~a.~%" time-of-day name) 
        (format t "What's up, ~a!~%" name)))

  (greet "Mr. Holmes")
  ;; Good morning, Mr. Holmes.
  (greet "Mr. Cumberbatch" t "evening")
  ;; Good evening, Mr. Cumberbatch.
  (greet "Mike" nil)
  ;; What's up, Mike!
\end{minted}
  \end{itemize}
\end{frame}

\begin{frame}[fragile]
  \frametitle{$\lambda$-lists 3/4}
  \begin{itemize}
  \item \textbf{Keyword parameters} are optional parameters that are specified
    in pairs $(name, value)$.
  \item Keyword parameters can be specified out of order and selectively.
  \item Common Lisp uses special type of symbols called ``keywords'' to specify
    names for the optional parameters.
  \item Keyword parameters are separated from the others in the $\lambda$-list
    with the \mintinline{Lisp}{&key} symbol:
\begin{minted}{Lisp}
  (defun greet (name &key (formally? t) (time-of-day "morning"))
    ...)

  (greet "Mr. Holmes" :time-of-day "afternoon")
  ;; Good afternoon, Mr. Holmes.
  (greet "Mike" :formally? nil)
  ;; What's up, Mike!
\end{minted}
  \end{itemize}
\end{frame}

\begin{frame}[fragile]
  \frametitle{$\lambda$-lists 4/4}
  \begin{itemize}
  \item \textbf{Rest-parameter} declaration in a $\lambda$-list allows to
    collect a variable number of arguments into a list.
  \item Rest-argument includes any optional and keyword arguments that were
    supplied to the function.
  \item Rest-argument can be declared with the \mintinline{Lisp}{&rest} symbol:
\begin{minted}{Lisp}
  (defun f (zero &rest args &key key1 key2 &allow-other-keys)
    (format t "zero: ~a, key1: ~a, key2: ~a~%" zero key1 key2)
    (format t "rest: ~a~%" args))

  (f 0 :key1 1 :key2 2 :key3 3 :key4 4)
  ;; zero: 0, key1: 1, key2: 2
  ;; rest: (:key1 1 :key2 2 :key3 3 :key4 4)
\end{minted}
  \item Allowed set of keyword arguments is limited by default, and must be
    relaxed with the \mintinline{Lisp}{&allow-other-keys} symbol.
  \end{itemize}
\end{frame}

\begin{frame}[fragile]
  \frametitle{$\lambda$-expressions (a.k.a. Anonymous Function)}
  \begin{itemize}
  \item \textbf{$\lambda$-expressions} are \textit{function literals} - then
    denote the functions without giving them names.
  \item Just regular function definitions, $\lambda$-expressions have two key
    components - $\lambda$-list and body. Unlike regular function definitions,
    $\lambda$-expressions do not have names.
  \item Compare:
\begin{minted}{Lisp}
  CL-USER> (defun f (x) (* x 2))
  f
  CL-USER> (f 5)
  10
\end{minted}
    and
\begin{minted}{Lisp}
  CL-USER> (lambda (x) (* x 2))
  #<Interpreted Function (unnamed) @ #x10008c1dbc2>
  CL-USER> ((lambda (x) (* x 2)) 5)
  10
\end{minted}
  \end{itemize}
\end{frame}

\begin{frame}[fragile]
  \frametitle{Functions as First-class Values 1/4}
  \begin{itemize}
  \item In programming languages, \textbf{first-class values} are values that
    can be \textit{passed as arguments to functions} and \textit{returned from
      functions as results}.
  \item Functions are fundamental mathematical objects, so in order to be
    ``functional'', programming language must consider functions
    \textit{first-class values}.
  \item Because Common Lisp is a Lisp-2, it needs a special operation to call
    functions from namespace 1:
\begin{minted}{Lisp}
  CL-USER> (let ((doubler (lambda (x) (* x 2))))
             (doubler 5))
  ;; Error: attempt to call `doubler' which is an undefined function.
  CL-USER> (let ((doubler (lambda (x) (* x 2))))
             (funcall doubler 5))
  10
\end{minted}
  \end{itemize}
\end{frame}

\begin{frame}[fragile]
  \frametitle{Functions as First-class Values 2/4}
  \begin{itemize}
  \item \mintinline{Lisp}{funcall} applies the function to the arguments as if
    its $\lambda$-list was the same as of the function being applied:
\begin{minted}{Lisp}
  CL-USER> (funcall #'+ 1 2 3 4)
  10
\end{minted}
  \item Sometimes it is more convenient to apply the function to the arguments
    collected into a list, which can be done with \mintinline{Lisp}{apply}:
\begin{minted}{Lisp}
  CL-USER> (apply #'+ '(1 2 3 4))
  10
\end{minted}
  \end{itemize}
\end{frame}

\begin{frame}[fragile]
  \frametitle{Functions as First-class Values 3/4}
  \begin{itemize}
  \item Passing functions as arguments:
\begin{minted}{Lisp}
  CL-USER> (defun twice (f x)
             (funcall f (funcall f x))
  twice
  CL-USER> (twice (lamdba (x) (* x x)) 5)
  125
\end{minted}
  \item Returning functions as results:
\begin{minted}{Lisp}
  CL-USER> (defun linear (a b)
             (lambda (x) (+ (* a x) b)))
  linear
  CL-USER> (let ((2x+5 (linear 2 5)))
             (funcall 2x+5 5))
  15
\end{minted}
  \end{itemize}
\end{frame}

\begin{frame}[fragile]
  \frametitle{Functions as First Class Values 4/4}
  \begin{itemize}
  \item Finally, having functions as first-class values allows to compose them
    to build more complex functions:
\begin{minted}{Lisp}
  (defun timed (f)
    (lambda (&rest args)
      (let* ((start-time (get-universal-time))
             (result (apply f args))
             (end-time (get-universal-time)))
        (format t "Elapsed time: ~a~%" (- end-time start-time))
        result)))

  (defun parser (string)
    ...)

  (funcall #'parser "<some string to be parsed>")

  (funcall (timed #'parser) "<some string to be parsed>")
\end{minted}
  \end{itemize}
\end{frame}

\begin{frame}
  \frametitle{Useful Resources}
  \begin{itemize}
  \item Common Lisp HyperSpec - publicly available alternative to ANSI Common
    Lisp standard
    \begin{itemize}
    \item \scriptsize{http://www.lispworks.com/documentation/HyperSpec/Front/index.htm}
    \end{itemize}
  \item Design Patterns in Dynamic Languages - Peter Norvig, 1998
    \begin{itemize}
    \item \scriptsize{https://norvig.com/design-patterns}
    \end{itemize}
  \end{itemize}
\end{frame}

\begin{frame}
  \frametitle{The End}
  \begin{center}
    Thank you!
  \end{center}
\end{frame}

\end{document}

%%% Local Variables:
%%% mode: latex
%%% TeX-master: t
%%% TeX-command-extra-options: "-shell-escape"
%%% End:
